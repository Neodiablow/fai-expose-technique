\documentclass[a4paper,12pt,one side,titlepage]{report}


%en francais
\usepackage[T1]{fontenc}
\usepackage{lmodern}
\usepackage[utf8]{inputenc}
\usepackage[francais]{babel}



\usepackage{listings}
\usepackage{geometry}
\usepackage{graphicx}
\usepackage{eurosym}
\usepackage{url}
\usepackage{pdfpages}
\usepackage[acronym]{glossaries}
\usepackage{hyperref}
%\usepackage[top=2.5cm,bottom=2.5cm,left=2.5cm,right=2.5cm]{geometry}
%\hypersetup{pdfborder=0}




\newglossaryentry{firewall}
{
	name={firewall},
	description={Protection pour serveur}
}

\newglossaryentry{naxsi}
{
	name={naxsi},
	description={Un type de firewall}
}



\makeglossaries
\begin{document}
%\includepdf[pages = {1-1}]{./pdf/1stpage.pdf}
%\includepdf[pages = {1-1}]{./pdf/pagegarde.pdf}


\begin{titlepage}
\begin{center}

\textsc{\LARGE Licence ASRALL}\\[1.5cm]

\textsc{\Large Exposé technique}\\[5.9cm]

% Title
{ \huge \bfseries Fully Automatic Installation\\[1.9cm] }

% Author and supervisor
\noindent
\begin{minipage}{0.4\textwidth}
\begin{flushleft} \large
François \textsc{Dupont}
\end{flushleft}
\end{minipage}%
\begin{minipage}{0.4\textwidth}
\begin{flushright} \large
Florent \textsc{Fillion}
\end{flushright}
\end{minipage}

\vfill

% Bottom of the page
{\large \today}

\end{center}
\end{titlepage}


%Page table des matières
\tableofcontents

%Introduction
\chapter{Introduction}
\section{Objectifs}
\subsection{Définition}
\textit{Fully Automatic Installation} ou \textsc{FAI} est un logiciel libre, inspiré de son équivalent \textsc{Solaris}, \textsc{Jumpstart}. Il permet d'installer et de configurer un système d'exploitation Linux sur une ou plusieurs machines, en utilisant un infrastructure client serveur, de façon rapide et automatisé.

Il est à noter que \textsc{FAI} n'est pas interactif contrairement à d'autres logiciels remplissant sensiblement la même fonction. Ce logiciel est également considéré comme mature (il en est à sa 4\textsuperscript{ème} itération majeure (4.0) et est développé de façon continue depuis 1999.

\subsection{Utilité}
\textsc{FAI} s'adresse particulièrement aux administrateurs ayant à gérer un grand parc de machine sous Linux, que ce parc soit \textit{virtualisé} ou physique (et même des \textit{chroot}s).

\section{Histoire}


\section{Concepts}
Cette partie pourrait avoir sa place dans le chapitre traitant de la technique à proprement parler, cependant, il nous semble essentiel d'exposer les principes fondamentaux (et simples une fois assimilés) de \textsc{FAI}.



\chapter{Technique}

\chapter{Alternatives}


\chapter{Sources}
http://fai-project.org/

\end{document}
